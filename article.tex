\documentclass[a4paper,10pt]{article}
\usepackage[utf8x]{inputenc}
% test for submission

%opening
\title{}
\author{}

\begin{document}

During the past few decades social network analysis has produced a great deal of insight into the workings of social systems. While social scientists have done a great deal of investigative work with regards to residential, friendship, trust, exchange or discussion networks, scientific inquiry has typically limited itself to investigating the characteristics of networks of only one kind. This approach has produced plentiful insight on the structure and function of different kinds of social networks, but the interaction between the different kinds of social networks has received insufficient investigation so far. Our work, in which we examine the interaction of residential and social networks represents an attempt at advancing this field of inquiry. More specifically, we extend a classic model of residential segregation (Schelling, 1968) by incorporating a social network that constructs -- and is influenced by -- residential preferences. We use Agent-Based Modelling to examine how social network topology affects residential segregation in the Schelling model. We start with a semi-static model, where social networks are fixed but the residential network is allowed to vary, after which we proceed to describe the case where both networks evolve in interdependence with each other. 

\section{The Static Model}

We start by fixing the social network, while allowing the residential network to vary. We concede that this model is far removed from social reality: nonetheless, examining this simplified case-study allows us to identify some patterns that may not become apparent in the dynamic version that we later present. As in Schelling's model, we start out with a set of agents who can have either of two colors. An agent's color is visible to all other agents with whom he or she comes into contact. The agents function in two different networks -- friendship and residential. We denote as ``friends'' an agent's connections in the friendship network, and as ``neighbors'' their connections in the residential network. The entire set of such connections will be known as the agent's ``neighborhood'' in the residential network, and as their friendship group in the friendship network. 

\subsection{Assumptions}

Our model copies Thomas Schelling's assumptions. We are concerned with the interactions between agents belonging to two distinct, non-overlapping, fixed groups. One agent's group membership is immediately visible to all other agents: in other words, there is no opportunity for dissimulation or ``passing.'' In practical terms, this means that none of our agents will be allowed to switch colors. Moreover, we assume that the two groups are socially salient to all actors: this excludes the possibility that an agent would have no opinion on the desirability / undesirability of one group over another, although they may have a neutral opinion on the subject. In the contemporary American context, for instance, skin color is an example of such a socially salient group, whereas eye color is not. Furthermore, we assume the existence of in-group bias, and of negligible ideological commitments to diversity. This final assumption underpins our specification of a lower threshold for the similarity of an actor and their neighbors (in a similar fashion to Schelling's model), without assuming an upper threshold for the same variable. 

\subsection{Mechanisms}

\paragraph{Residential Mobility} Our residential mobility mechanism copies the one Schelling proposed in his classic 1968 paper. At every iteration of the model, each actor $A_i$ has a tolerance level, $\tau_i$, indicating the maximum number of dissimilar neighbors the actor would be comfortable with. The actor compares $\tau_i$ with the realized distribution of neighbors in their Moore neighborhood\footnote{A Moore neighborhood comprises one's immediate eight neighbors on a lattice, accounting for adjacent areas on the horizontal, vertical, as well as diagonal axes.}. If the realized number of dissimilar neighbors is lower or equal than $\tau_i$, the actor will be content with their current residential situation. If, on the contrary, the realized number of dissimilar neighbors is greater than $\tau_i$, the actor will become unhappy with their current situation, and will decide to move. As in Schelling's model, an actor will only be allowed to move if there is a free (``vacant'') square next to their current residence. While this requirement contradicts the reality of the housing market, it has the useful property of creating situations where unhappy actors may become ``trapped'' in a neighborhood. 

The significance of this feature will become apparent in our discussion of the dynamic extension of the model, where we allow the possibility of new friendship ties to arise between neighbors. In this case, if an actor wants to move (because of ``unhappiness'' as defined above), but cannot find an empty square, they will be forced to stay in the neighborhood independent of their will. The longer the actor remains in the neighborhood without being able to move, the higher the likelihood that he or she will become friends with their neighbors, who are more diverse than the actor would have initially wanted. But the more diverse friends the actor acquires, the lower their threshold for similarity begins. In these conditions, given enough time, an actor will become happy if they live in a diverse neighborhood. This mechanism parallels the so-called ``contact hypohtesis'' [REFERENCES], according to which inter-group contact reduces prejudice. Even though it starts from an unrealistic assumption this mechanism is analogous to a realistic situation which would take many more variables to reproduce exactly. People do not become trapped in a neighborhood because they cannot move in the adjacent house: there are however many situations (insufficient resource, jobs, family ties etc.) that keep them in a neighborhood even when they would rather move. In this case the potential event of not finding a vacancy nearby approximates the possibility that an actor may not be able to move for myriad other reasons, a parsimonious model of which would be impossible to develop. 

 

\paragraph{Social Network Influence} is the second mechanism on which our model is based. While determining the exact mechanics of the process as they unfold in the real world is beyond the scope of our paper, we devote our attention to choosing a theoretically useful, if simplified, mechanism of social influence. A naive observer might posit that an actor  simply look at their immediate friendship circle, weighing each friend's information inversely proportionally to one's total number of friends. Alas, such an interpretation is completely ignorant of network topology and, more broadly, of status dynamics at work in the friendship network, if we assume status to be reflected in one's degree in the friendship network. \textit{Ceteris paribus,} a high-status friend should have a higher influence than a lower-status one. Similarly, the influence of a friend should be lower if the friend is lower-status compared to the actor, and higher if the friend is higher-status when compared to the actor. Thus we normalize actor $X$'s weighting of friend's $Y$'s information by $\frac{deg(Y)}{\mathrm{max.deg.}} \times \frac{deg(Y)}{deg(X)}$, where $deg(X)$ and $deg(Y)$ represent the degrees of the two actors, while $\mathrm{max.deg.}$ is the maximum degree of any node in the network. The normalization reflects both global and local phenomena: whereas the former factor accounts for status in the context of the entire network, the latter adjusts for dyadic status disparities.

A more substantive issue has to do with what kind of information is communicated through the social network. To the extent that preferences for or against residential segregation made the object of frequent discussion, one could argue that the social network acts as a conduit for information about each actor's threshold, which the actor broadcasts to their friends. This assumption would involve the possibility that intolerant preferences be freely transmissible in the network. There is ample literature that shows this not to be the case, however. Already in 1982, Pettigrew noted a steep decline in Americans' explicitly racist attitudes, and so it seems highly unlikely that Americans' would communicate their preferences for segregation explicitly to each other. Moreover, an entire literature has developed in social psychology on the effects of implicit, sub-conscious intergroup bias. [CITATIONS; I.e., Blair and Banaji, etc.]. Thus explicit communication seems too unlikely a process to be useful for our model. Instead we propose a mechanism whereby actors simply take stock of their friends' ``color,'' weighted according to the normalization mechanism proposed above, and adjust their tolerance levels accordingly. Having many friends belonging to the other group should make one more tolerant - but as mentioned above, the effect should be lower if one's friends are low-status, and even lower when the actor is high-status.

\subsection{Measuring Segregation}

To maintain comparability 

\paragraph{Independent Variables}

\end{document}
